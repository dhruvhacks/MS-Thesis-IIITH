\chapter{Introduction}
\label{ch:intro}

Movies are a powerful medium for storytelling, engaging audiences through a rich collection of characters and emotions. Understanding the intricacies of characters' emotions and mental states within movie narratives is essential for unraveling the layers of storytelling and enhancing our grasp of cinematic experiences. This thesis embarks on a journey into the realm of movie story analysis, specifically focusing on the nuanced task of character emotion and mental state prediction.

The ability to decipher the emotional nuances of characters in movies extends beyond mere entertainment; it opens avenues for exploring the psychological depths of storytelling, character development, and audience engagement. This research takes a significant step forward by formulating emotion understanding as a predictive task, not only at the level of entire movie scenes but also at the individual character level. Our primary focus is on predicting a diverse and multi-label set of emotions, ranging from classic emotions like happiness and anger to more intricate mental states such as honesty and helpfulness.

To address this challenge, we introduce EmoTx \cite{dhruv2023emotx}, a novel multimodal Transformer-based architecture that harnesses the combined power of video data, multiple characters, and dialogues. This architecture facilitates joint predictions for scene and character emotions and mental states, allowing for a holistic understanding of the emotional landscape within movie scenes. Leveraging annotations from the MovieGraphs dataset~\cite{moviegraphs}, our model is trained to capture the spectrum of emotions and mental states, providing a comprehensive and granular analysis.
